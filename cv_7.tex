%%%%%%%%%%%%%%%%%%%%%%%%%%%%%%%%%%%%%%%%%
% "ModernCV" CV and Cover Letter
% LaTeX Template
% Version 1.11 (19/6/14)
%
% This template has been downloaded from:
% http://www.LaTeXTemplates.com
%
% Original author:
% Xavier Danaux (xdanaux@gmail.com)
%
% License:
% CC BY-NC-SA 3.0 (http://creativecommons.org/licenses/by-nc-sa/3.0/)
%
% Important note:
% This template requires the moderncv.cls and .sty files to be in the same 
% directory as this .tex file. These files provide the resume style and themes 
% used for structuring the document.
%
%%%%%%%%%%%%%%%%%%%%%%%%%%%%%%%%%%%%%%%%%

%----------------------------------------------------------------------------------------
%	PACKAGES AND OTHER DOCUMENT CONFIGURATIONS
%----------------------------------------------------------------------------------------

\documentclass[11pt,a4paper,sans]{moderncv} % Font sizes: 10, 11, or 12; paper sizes: a4paper, letterpaper, a5paper, legalpaper, executivepaper or landscape; font families: sans or roman
\usepackage[utf8]{inputenc}

\moderncvstyle{classic} % CV theme - options include: 'casual' (default), 'classic', 'oldstyle' and 'banking'
\moderncvcolor{blue} % CV color - options include: 'blue' (default), 'orange', 'green', 'red', 'purple', 'grey' and 'black'

\usepackage{lipsum} % Used for inserting dummy 'Lorem ipsum' text into the template

\usepackage[scale=.85]{geometry} % Reduce document margins
%\setlength{\hintscolumnwidth}{3cm} % Uncomment to change the width of the dates column
%\setlength{\makecvtitlenamewidth}{10cm} % For the 'classic' style, uncomment to adjust the width of the space allocated to your name

%----------------------------------------------------------------------------------------
%	NAME AND CONTACT INFORMATION SECTION
%----------------------------------------------------------------------------------------

\firstname{Fátima L. S.} % Your first name
\familyname{Nunes} % Your last name

% All information in this block is optional, comment out any lines you don't need
\title{Curriculum Vitae}
\address{University of São Paulo}{Brazil}
\email{fatima.nunes@usp.br}
\homepage{http://lapis.each.usp.br} {http://lapis.each.usp.br} % The first argument is the url for the clickable link, the second argument is the url displayed in the template - this allows special characters to be displayed such as the tilde in this example
% \extrainfo{DOB: Month Day, Year}
\photo[70pt][0.4pt]{pictures/foto} % The first bracket is the picture height, the second is the thickness of the frame around the picture (0pt for no frame)
%\quote{"A witty and playful quotation" - John Smith}
%----------------------------------------------------------------------------------------

\begin{document}

\makecvtitle % Print the CV title
%----------------------------------------------------------------------------------------
%	EDUCATION SECTION
%----------------------------------------------------------------------------------------
%Full Professor at the University of São Paulo. Bachelor in Computer Science (Universidade Estadual Paulista Julio de Mesquita), Master in Electrical Engineering (University of São Paulo) and PhD in Science (Computational Physics) (University of São Paulo). Her research interests include Virtual Reality, Image Processing, Content-based multimedia data retrieval. Several of her works are applied in healthcare, establishing interfaces with the of Biomedical Engineering area.

\section{Education}

\cventry{1997--2001}{PhD in Science}{University of São Paulo}{}{\textit{Computational Physics}}{}

\cventry{1995--1997}{ Master in Electrical Engineering}{University of São Paulo}{}{}{}

\cventry{1997--2001}{Bachelor in Computer Science}{Universidade Estadual Paulista Julio de Mesquita}{}{}{}

%----------------------------------------------------------------------------------------
%	AWARDS SECTION
%----------------------------------------------------------------------------------------
\section{Positions}

\cventry{2017--Present}{Full Professor}{University of São Paulo}{}{}{}

\cventry{2013--2017}{Associate Professor}{University of São Paulo}{}{}{}

\cventry{2008--2013}{Assistant Professor}{University of São Paulo}{}{}{}

\cventry{2002--2008}{Assistant Professor}{Centro Universitário Euripedes de Marília}{}{}{}


%----------------------------------------------------------------------------------------
%	Skills SECTION
%----------------------------------------------------------------------------------------

\section{Teaching}

\cvitem{Programming Languages}{\textsc{LabVIEW, MATLAB}}
\cvitem{CAD}{\textsc{SOLIDWORKS}}
\cvitem{Software }{\textsc{LaTeX, MS Word, MS Excel}}

%----------------------------------------------------------------------------------------
%	Masters SECTION
%----------------------------------------------------------------------------------------

%\section{Masters Thesis}

%\cvitem{Title}{\emph{Technologies and characterization of ferroelectric polymers for biomedical sensors}}
%\cvitem{Supervisors}{Professor Antonino Fiorillo}
%\cvitem{Description}{This thesis is based on the implementation of a temperature sensor.}

%----------------------------------------------------------------------------------------
%	WORK EXPERIENCE SECTION
%----------------------------------------------------------------------------------------

\section{Selected Journal Publications}
\footnotesize
\begin{enumerate}
    \item BERGAMASCO, LEILA CRISTINA C. ; NUNES, FÁTIMA L.S. . Intelligent retrieval and classification in three-dimensional biomedical images - A systematic mapping. Computer Science Review, v. 31, p. 19-38, 2019.
    \item CORRÊA, C. G. ; NUNES, FATIMA L.S. ; RANZINI, E. ; NAKAMURA, R. ; TORI, R. . Haptic interaction for needle insertion training in medical applications: The state-of-the-art. MEDICAL ENGINEERING \& PHYSICS, v. 63, p. 6-25, 2019.
    \item PEDRO, RICARDO WANDRÉ DIAS ; MACHADO-LIMA, ARIANE ; NUNES, FÁTIMA L.S. . Is mass classification in mammograms a solved problem? - A critical review over the last 20 years. EXPERT SYSTEMS WITH APPLICATIONS, v. 119, p. 90-103, 2019.
    \item ARANHA, RENAN VINICIUS ; CORREA, CLEBER GIMENEZ ; NUNES, FATIMA L. S. .Adapting software with Affective Computing: a systematic review. IEEE Transactions on Affective Computing, v. 1, p. 1-1, onic. 2019.
    \item TESTA, RAFAEL LUIZ ; CORRÊA, CLÉBER GIMENEZ ; MACHADO-LIMA, ARIANE ; NUNES, FÁTIMA L. S. . Synthesis of Facial Expressions in Photographs. ACM COMPUTING SURVEYS, v. 51, p. 1-35, 2019.
    \item MASSETTI, THAIS ; FÁVERO, FRANCIS MEIRE ; MENEZES, LILIAN DEL CIELLO DE ; ALVAREZ, MAYRA PRISCILA BOSCOLO ; CROCETTA, TÂNIA BRUSQUE ; GUARNIERI, REGIANI ; NUNES, FÁTIMA L. S. ; MONTEIRO, CARLOS BANDEIRA DE MELLO ; SILVA, TALITA DIAS DA . Achievement of Virtual and Real Objects Using a Short-Term Motor Learning Protocol in People with Duchenne Muscular Dystrophy: A Crossover Randomized Controlled Trial. Games for Health Journal, v. 7, p. 1-9, 2018.
    \item CORRÊA SOUZA, ALINNE C. ; NUNES, FÁTIMA L. S. ; DELAMARO, MÁRCIO E. . An automated functional testing approach for virtual reality applications. SOFTWARE TESTING VERIFICATION \& RELIABILITY, v. 8, p. e1690, 2018.
    \item RODRIGUES, D. S. ; DELAMARO, MÁRCIO E. ; CORRÊA, C. G. ; NUNES, FATIMA L.S. . Using Genetic Algorithms in Test Data Generation. ACM COMPUTING SURVEYS, v. 51, p. 1-23, 2018.
    \item CORRÊA, C. G. ; MACHADO, M. A. A. M. ; RANZINI, E. ; TORI, R. ; NUNES, F. L. S. . Virtual Reality simulator for dental anesthesia training in the inferior alveolar nerve block. Journal of Applied Oral Science, v. 25, p. 357-366, 2017.
    \item NUNES, EUNICE P. DOS SANTOS ; ROQUE, LICINIO G. ; NUNES, FATIMA DE LOURDES DOS SANTOS . Measuring Knowledge Acquisition in 3D Virtual Learning Environments. IEEE Computer Graphics and Applications, v. 36, p. 58-67, 2016.
    \item BISCARO, H. H. ; SANTOS, J. ; PEREIRA, G. R. ; NUNES, F. L. S. . Comparing efficient data structures to represent geometric models for three-dimensional virtual medical training. Journal of Biomedical Informatics, v. 63, p. 195-211, 2016.
    \item RIBEIRO, M. L. ; LEDERMAN, H. M. ; ELIAS, S ; NUNES, FÁTIMA L. S. . Techniques and Devices Used in Palpation Simulation with Haptic Feedback. ACM COMPUTING SURVEYS, v. 49, p. 1-28, 2016.
    \item PEDRO, R. W. D. ; NUNES, F. L. S. ; MACHADO-LIMA, A. . Using grammars for pattern recognition in images: a systematic review. ACM Computing Surveys, v. 46, p. 1-34, 2013.
    \item DELAMARO, MARCIO EDUARDO ; DE LOURDES DOS SANTOS NUNES, FÁTIMA ; DE OLIVEIRA, RAFAEL ALVES PAES . Using concepts of content-based image retrieval to implement graphical testing oracles. Software Testing, Verification \& Reliability, v. 23, p. 171-198, 2013.
    \item OLIVEIRA, ANA CLÁUDIA MELO TIESSI GOMES ; SANTOS NUNES, FÁTIMA DE LOURDES . Building a Open Source Framework for Virtual Medical Training. Journal of Digital Imaging, v. 23, p. 706-720, 2010.
    \item NUNES, F. L. S.; SCHIABEL, H. ; GÓES, C. E. . Contrast enhancement in dense breasts images to aid clustered microcalcifications detection. Journal of Digital Imaging, v. 20, p. 53-66, 2007.
    \item NUNES, F. L. S.; SCHIABEL, H. ; BENATTI, R. H. . Contrast Enhancement in dense breast images by using modulation transfer function. Medical Physics (Lancaster), Estados Unidos, v. 29(12), p. 2925-2936, 2002.
    \item GÓES, C. E. ; SCHIABEL, H. ; NUNES, F. L. S. . Evaluation of Microcalcifications Segmentation techniques for Dense Breast digitized images. Journal of Digital Imaging, Cleveland, v. 15, n.1, p. 231-233, 2002.
    \item SCHIABEL, H. ; NUNES, F. L. S. ; ESCARPINATI, M. ; GÓES, C. E. . Investigations on the effect of different characteristics of images sets on the performance of a processing scheme for microcalcifications detection in digital mammograms.. Journal of Digital Imaging, Salt Lake City, USA, v. 14, n.2, p. 224-225, 2001.
    \item NUNES, F. L. S.; SCHIABEL, H. ; ESCARPINATI, M. ; BENATTI, R. H. . Comparisons of different contrast resolutions effects on a computer-aided detection system intended to clustered microcalcifications detection in dense breasts images.. Journal of Digital Imaging, Salt Lake City, v. 14, n.2, p. 217-219, 2001.
    \item VIEIRA, M. A. C. ; ESCARPINATI, M. ; SCHIABEL, H. ; CAETANO, C. A. C. ; NUNES, F. L. S. . A Segmentation Technique for Digitized Mammograms: Saving Processing Time and Memory. Medical \& Biological Engineering \& Computing, v. 39-2, p. 558-561, 2001.
\end{enumerate}

\section{Books organization}
\footnotesize
\begin{enumerate}
    \item NUNES, F. L. S.; MACHADO, L. S. (Org.) ; PINHO, M. S. (Org.) ; KIRNER, C. (Org.) . Abordagens práticas de realidade virtual e aumentada. 1. ed. Porto Alegre (RS): Sociedade Brasileira de Computação, 2009. v. 1. 1p .
    \item TEICHRIEG, V. (Org.) ; NUNES, F. L. S. (Org.) ; MACHADO, L. S. (Org.) ; TORI, R. (Org.) . Realidade Virtual e Aumentada na prática. 1. ed. João Pessoa (PB): Sociedade Brasileira de Computação, 2008. v. 1. 164p .
\end{enumerate}

\section{Book chapters}
\footnotesize
\begin{enumerate}
    \item NUNES, FATIMA L.S.; TREVISAN, D. G. ; NUNES, E. P. S. ; CORRÊA, C. G. ; SANCHES, S. R. R. ; DELAMARO, MÁRCIO E. ; SANTOS, A. C. C. . Avaliação. In: Tori, Romero; Hounsell, Marcelo da Silva. (Org.). Introdução a Realidade Virtual e Aumentada. 1ed.Porto Alegre (RS): Sociedade Brasileira de Computação, 2018, v. 1, p. 1-15.
    \item NUNES, FATIMA L.S.; OLIVEIRA, J. C. ; MACHADO, L. S. ; NUNES, E. P. S. ; COSTA, R. M. E. M. ; QUEIROZ, A. C. M. . Saúde. In: Tori, Romero; Hounsell, Marcelo da Silva.. (Org.). Introdução a Realidade Virtual de Aumentada. 1ed.Porto Alegre (RS): Sociedade Brasileira de Computação, 2018, v. 1, p. 16-30.
    \item CARDOSO, A. ; LAMOUNIER JR, E. ; ZORZAL, Ezequiel Roberto ; NUNES, F. L. S. ; MIRANDA NETO, M. ; PROENCA, A. P. . Realidade Virtual e Aumentada na Saúde e Reabilitação. In: Adriano de Oliveira Andrade; Alcimar Barbosa Soares; Alexandre Cardoso; Edgard Afonso Laumounier. (Org.). Tecnologias, técnicas e tendências em Engenharia Biomédica. 1ed.Bauru: Canal6, 2014, v. 1, p. 402-416.
    \item MORAES, R. M. ; MACHADO, L. S. ; NUNES, F. L. S. ; COSTA, R. M. E. M . Serious Games and Virtual Reality for Education, Training and Health. In: Maria Manuela Cruz-Cunha. (Org.). Handbook of Research on Serious Games as Educational, Business and Research Tools. 1ed.: IGI Global, 2012, v. 1, p. 315-336.
    \item OLIVEIRA, J. C. ; SANTOS, S. R. ; CARVALHO, B. M. ; MACHADO, L. S. ; MORAES, R. M. ; COSTA, R. M. E. M ; WERNECK, V. M. B. ; PINHO, M. S. ; NUNES, F. L. S. ; PEDRINI, H. ; RODELLO, I. A. ; DELAMARO, M. E. ; TORI, R. . Virtual Reality Applied to Medicine. In: Feijóo, R. A.; Ziviani, A.; Blanco, P. J. (Org.). Scientific Computing Applied to Medicine and Healthcare. Scientific Computing Applied to Medicine and Healthcare. 1ed.Petrópolis (RJ): INCT-MACC/LNCC, 2012, v. 1, p. 276-314.
    \item NUNES, F. L. S.; COSTA, R. M. E. M ; MACHADO, L. S. ; MORAES, R. M. . Desenvolvendo aplicações de RVA para saúde: imersão, realismo e motivação. In: Ribeiro, M. W. S.; Zorzal, E.R.. (Org.). Realidade Virtual e Aumentada: aplicações e tendências. 1ed.Uberlândia (MG): Sociedade Brasileira de Computação, 2011, v. 1, p. 81-94.
    \item NUNES, F. L. S.; DELAMARO, M. E. . Recuperação de imagens baseada em conteúdo e sua aplicação na área de saúde. In: Anita Maria da Rocha Fernandes, Michelle Silva Wangham. (Org.). Computer on the Beach 2010 - Livro de minicursos. 1ed.Florianópolis: , 2010, v. 1, p. 115-144.
    \item NUNES, F. L. S.; CORRÊA, C. G. . Interação com Java3D. In: José Remo Ferreira Brega; Judith Kelner. (Org.). Interação com Realidade Virtual e Realidade Aumentada. 1ed.Bauru (SP): Canal 6, 2010, v. 1, p. 105-118.
    \item MACHADO, L. S. ; MORAES, R. M. ; NUNES, F. L. S. . Serious games para saúde e treinamento imersivo. In: Nunes, F.L.S.; Machado,L.S.;Pinho, M.S.; Kirner,C.. (Org.). Abordagens práticas de realidade virtual e aumentada. 1ed.Porto Alegre (RS): Sociedade Brasileira de Computação, 2009, v. 1, p. 31-60.
    \item CORRÊA, C. G. ; NUNES, F. L. S. . Interação com dispositivos convencionais e não convencionais utilizando integração entre linguagens de programação. In: Nunes, F.L.S.;Machado,L.S.;Pinho, M.S.; Kirner, C.. (Org.). Abordagens práticas de realidade virtual e aumentada. 1ed.Porto Alegre (RS): Sociedade Brasileira de Computação, 2009, v. 1, p. 61-103.
    \item NUNES, F. L. S.; MACHADO, L. S. ; COSTA, R. M. E. M . Realidade Virtual e Realidade Aumentada aplicadas à Saúde. In: Costa, R.M.; Ribeiro, M. W.S.. (Org.). Aplicações de Realidade Virtual e Aumentada. 1ed.Porto Alegre (RS): Sociedade Brasileira de Computação, 2009, v. 1, p. 69-89.
    \item NUNES, F. L. S.; BALANIUK, R. . Realidade Virtual aplicada a saúde - conceitos e situação atual. In: Lourdes Mattos Brasil. (Org.). Informática em Saúde. 1ed.Brasília/Londrina: Editora Universa (UCB) / Editora da Universidade Estadual de Londrina, 2008, v. 1, p. 325-355.
    \item NUNES, F. L. S.; COSTA, R. M. E. M . RV para a área médica: requisitos, dispositivos e implementação. In: Veronica Teichrieb, Fátima de Lourdes dos Santos Nunes, Liliane dos Santos Machado, Romero Tori. (Org.). Realidade Virtual e Aumentada na prática. 1ed.João Pessoa (PB): Sociedade Brasileira de Computação, 2008, v. 1, p. 119-146.
    \item NUNES, F. L. S.; COSTA, R. M. E. M ; OLIVEIRA, A. C. M. T. G. ; DELFINO, S. R. ; PAVARINI, L. ; RODELLO, I. A. ; BREGA, J. R. F. ; SEMENTILLE, A. C. . Aplicações Médicas usando Realidade Virtual e Realidade Aumentada. In: Claudio Kirner; Robson Siscoutto. (Org.). Realidade Virtual e Aumentada - Conceitos, Projeto e Aplicações. 1ed.Porto Alegre: Editora SBC, 2007, v. 1, p. 223-255.
    \item RODELLO, I. A. ; SEMENTILLE, A. C. ; BREGA, J. R. F. ; NUNES, F. L. S. . Sistemas Distribuídos de Realidade Virtual e Aumentada. In: Claudio Kirner; Robson Siscoutto. (Org.). Realidade Virtual e Aumentada - Conceitos, Projeto e Aplicações. 1ed.Porto Alegre: Editora SBC, 2007, v. 1, p. 129-150.
    \item NUNES, F. L. S.. Introdução ao Processamento de Imagens Médicas para Auxílio ao Diagnóstico. In: Karin Breitman; Ricardo Anido. (Org.). Atualizações em Informática. 1ed.Rio de Janeiro: PUC-Rio, 2006, v. 1, p. 73-126.
    \item NUNES, F. L. S.. Processamento de imagens médicas para sistemas de auxílio ao diagnóstico. In: Ildeberto Aparecido Rodello, José Remo Ferreira Brega, Kalinka R. L. J. Castelo Branco. (Org.). Escola Regional de Informática São Paulo / Oeste 2006. 1ed.Marília (SP): Estrela, 2006, v. 1, p. 83-137.
\end{enumerate}

\section{Selected Conferences \& workshops Publications}
\footnotesize
\begin{enumerate}
    \item PEDRO, RICARDO WANDRÉ DIAS ; MACHADO-LIMA, A. ; NUNES, FATIMA L.S. A new syntactic approach for masses classification in digital mammograms. In: IEEE 32nd International Symposium on Computer-Based Medical Systems (CBMS), 2019, Córdoba, Espanha. 
    \item Bergamasco, Leila C. C. ; ROCHITTE, C. E. ; NUNES, FATIMA L.S. . 3D medical objects processing and retrieval using Spherical Harmonics: a case study with Congestive Heart Failure MRI exams. In: The 33rd ACM/SIGAPP Symposium On Applied Computing, 2018, Pau, France. Proceedings of the 33rd ACM/SIGAPP Symposium On Applied Computing. Pau, France: ACM SIGAPP, 2018. v. 1. p. 26-33.
    \item Bergamasco, Leila C. C. ; NUNES, FATIMA L.S. ; LIMA, K. ; ROCHITTE, C. E. . 3D medical objects retrieval approach using SPHARMs descriptor and network flow as similarity measure. In: 31st Conference on Graphics, Patterns and Images (SIBGRAPI 2018), 2018, Foz do Iguaçu (PR). Proceedings of the 31st Conference on Graphics, Patterns and Images (SIBGRAPI 2018), 2018. v. 1. p. 1-8.
    \item Testa, Rafael Luiz, Ariane Machado Lima, and Fátima de Lourdes dos Santos Nunes. "Factors influencing the perception of realism in synthetic facial expressions." 2018 31st SIBGRAPI Conference on Graphics, Patterns and Images (SIBGRAPI). IEEE, 2018.
    \item NUNES, F. L. S.; Bergamasco, Leila C. C. ; DELMONDES, P. H. ; VALVERDE, M. A. G. ; JACKOWSKI, M. P. . A Novel 3D Shape Descriptor for automatic Retrieval of Anatomical Structures from Medical Images. In: SPIE Medical Imaging, 2017, Orlando, FL, EUA. Proc. SPIE 10134, Medical Imaging 2017: Computer-Aided Diagnosis, 1013430 (March 3, 2017); doi:10.1117/12.2253928. Orlando, FL, EUA: SPIE, 2017. v. 1.
    \item MUNIZ, FREDERICO B. ; ARAÚJO, LUCIANO V. ; NUNES, FÁTIMA L. S. . A cloud collaborative medical image platform oriented by social network. In: SPIE Medical Imaging, 2017, Orlando. v. 1. p. 101380X-6.
    \item ARANHA, RENAN V. ; SILVA, L. S. ; CHAIM, M. L. ; NUNES, FÁTIMA L. S. . Using Affective Computing to automatically adapt serious games for rehabilitation. In: IEEE 30th International Symposium on Computer-Based Medical Systems, 2017, Thessaloniki. Proceedings of the IEEE 30th International Symposium on Computer-Based Medical Systems. Piscataway, NJ, EUA: IEEE Computer Society, 2017. v. 1. p. 55-50.
    \item CORRÊA, C. G. ; TOKUNAGA, D. M. ; RANZINI, E. ; NUNES, FÁTIMA L. S. ; TORI, R. . Haptic Interaction Objective Evaluation in Needle Insertion Task Simulation. In: ACM Symposium on Applied Computing - ACM SAC, 2016, Pisa (Itália). Proceeding of the 2016 ACM Symposium on Applied Computing. ACM: ACM, 2016. v. 1. p. 149-154.
    \item TESTA, R. L. ; MUNIZ, A. H. N. ; CARPIO, L. U. S. ; DIAS, R. S. ; ROCCA, C. C. A. ; MACHADO-LIMA, A. ; NUNES, F. L. S. . Generating facial emotions for diagnosis and training. In: 28th IEEE International Symposium on Computer-Based Medical Systems - CBMS, 2015, São Carlos (SP). Proceedings of the 28th IEEE International Symposium on Computer-Based Medical Systems - CBMS, 2015. v. 1. p. 304-309.
\end{enumerate}
%------------------------------------------------

\section{Work Experience}

\cventry{Dec 2015 --
Present}{Assistant Engineer}{\textsc{Integrated Motions, Inc}}{Berkeley}{Ca}{Contracted with ChemiSense to characterize and calibrate proprietary air quality sensors.  }

\cventry{June 2015 --
Present}{Undergraduate Researcher}{\href{http://best.berkeley.edu/best-research/best-berkeley-emergent-space-tensegrities-robotics/}{\textsc{UC Berkeley Emergent Space Tensegrities Lab}}}{}{}{Collaborating in the design, manufacturing, and assembling of a tensegrity robot.  Designed and manufactured an apparatus to test the friction in the end caps.  Helped design new end caps that reduced friction.  Working to create a 3-D model that mimics the orientation of the robot using sensor readings.}

\cventry{June 2015 --
Dec 2015}{Independent Contractor}{\textsc{California Department of Public Health}}{Richmond}{Ca}{Used wet and dry air to regulate the relative humidity in an environmental chamber.  Designed and implemented a closed-loop controller in LabVIEW.  Setup a Data Acquisition Module (DAQ) to take sensor readings and manage multiple mass flow controllers.}

\cventry{March 2007 --
June 2012}{Transportation Security Officer}{\textsc{Covenant Aviation Security, LLC}}{San Francisco}{Ca}{Thoroughly screened passengers using a variety of new technology to protect all passengers while maintaining a high level of professionalism with excellent customer service.  Promoted as a recruitment team member to lead the hiring process.  This included reviewing resumes, conducting interview, and helping new hires with the on-boarding process.  Successfully completed a Professional Development program aimed at developing managerial skills.}

\cventry{Aug 2006 --
March 2007}{Leasing Agent}{\textsc{Braddock and Logan Services, LLC}}{Danville}{Ca}{Provided superior customer service and communication to high-end residents and prospects.  Maximized customer satisfaction and increased renewals, revenue, reputation and profitability.  Assisted in maintenance, lease administration, eviction, vacancy anticipation, collection, marking, lease renewals and audits.  Prepared weekly and monthly rental statistic reports and market rent surveys.}

\cventry{Oct 2005 --
Aug 2006}{Tire Installer}{\textsc{Costco Wholesale}}{Livermore}{Ca}{Drove customer vehicles into/out of installation bay and raised vehicles using a hydraulic lift.  Mounted and dismounted tires from the rims and balanced them using a balancing machine.  Assisted members in determining the correct tires needed for their vehicle.  Ensured tires were properly inflated and lug nuts were properly tightened.  Repaired flat tires.}

\cventry{Aug 2005 --
Oct 2005}{Front Desk Receptionist}{\textsc{24 Hour Fitness}}{Fremont}{Ca}{Greeted all incoming members and guests and directed them appropriately.  Operated a cash register and sold merchandise to members.  Maintained logs of maintenance requests and member concerns.  Performed general cleaning duties.}

\cventry{Feb 2004 --
Aug 2005}{Desktop Support Technician}{\textsc{CafePress.com}}{Foster City}{Ca}{Supported two locations with 200+ employees.  Diagnosed hardware and software issues and handled new equipment roll-out.  Prepared reports for hardware and software auditing and also license and asset tracking.  Administrated the mail server, phone system, alarm system, and building access.}

\cventry{Nov 2003 --
Feb 2004}{Seasonal Sales Associate}{\textsc{Macy's}}{Concord}{Ca}{Provided excellent customer service throughout the transaction process.  Provided assistance in the selection of merchandise to purchase.  Continually met personal sales goals and member reward card enrollment goals.  Processed returns courteously and professionally.}

\cventry{June 2003 --
Aug 2003 \& June 2002 -- Jan 2003}{IT Technician}{\textsc{Eskanos and Adler, Lawfirm Collections Agency}}{Concord}{Ca}{Provided IT support for 140+ users and was the primary contact for the call center employees.  Troubleshot computer and network issues as well as software and hardware upgrades.  Built new hire workstations.  Created an inventory of computer assets and a method for tracking new and retired computer parts.}


%----------------------------------------------------------------------------------------
%	Coursework SECTION
%----------------------------------------------------------------------------------------

\section{Coursework}

\renewcommand{\listitemsymbol}{-~} % Changes the symbol used for lists

\cvlistdoubleitem{Thermodynamics}{Heat Transfer}
\cvlistdoubleitem{Combustion Processes}{Fluid Mechanics}
\cvlistdoubleitem{Introduction to Mechanical Systems for Mechatronics}{Design of Microprocessor Based Mechanical Systems}
\cvlistdoubleitem{Mechanical Engineering Laboratory}{Introduction to Solid Mechanics}
\cvlistdoubleitem{Mechanical Behavior of Engineering Materials}{Dynamic Systems and Feedback Control}
\cvlistdoubleitem{Mechatronics Design}{Feedback Control Systems}

%----------------------------------------------------------------------------------------
%	COMMUNICATION SKILLS SECTION
%----------------------------------------------------------------------------------------

%\section{Communication Skills}

%\cvitem{2010}{Oral Presentation at the California Business Conference}
%\cvitem{2009}{Poster at the Annual Business Conference in Oregon}

%----------------------------------------------------------------------------------------
%	LANGUAGES SECTION
%----------------------------------------------------------------------------------------

\section{Languages}

\cvitemwithcomment{English}{Native Speaker}{}
%\cvitemwithcomment{Spanish}{Basic}{}
%\cvitemwithcomment{Dutch}{Basic}{Basic words and phrases only}

%----------------------------------------------------------------------------------------
%	INTERESTS SECTION
%----------------------------------------------------------------------------------------

% \section{Interests}

% \renewcommand{\listitemsymbol}{-~} % Changes the symbol used for lists
% \cvlistdoubleitem{Piano}{}
% \cvlistitem{Baseball}

%----------------------------------------------------------------------------------------
%	COVER LETTER
%----------------------------------------------------------------------------------------

% To remove the cover letter, comment out this entire block

%\clearpage

%\recipient{HR Department}{Corporation\\123 Pleasant Lane\\12345 City, State} % Letter recipient
%\date{\today} % Letter date
%\opening{Dear Sir or Madam,} % Opening greeting
%\closing{Sincerely yours,} % Closing phrase
%\enclosure[Attached]{curriculum vit\ae{}} % List of enclosed documents

%\makelettertitle % Print letter title

%\lipsum[1-3] % Dummy text

%\makeletterclosing % Print letter signature

%----------------------------------------------------------------------------------------

\end{document}



